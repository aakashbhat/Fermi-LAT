\documentclass[preprint]{aa}
\pdfoutput=1

\bibliographystyle{aa}
%\AuthorCallLimit=200
\usepackage{booktabs}
\usepackage{csvsimple}
\usepackage{pgfplotstable}
\usepackage{array}
%\usepackage{colortbl}
\usepackage{amsmath}
\usepackage{amsfonts}
\usepackage{dsfont}
\usepackage{amsxtra}
\usepackage{hyperref}
\usepackage{amssymb}
\usepackage{upgreek}
\usepackage{comment}

\usepackage{multirow}
\usepackage{url}

\usepackage{graphicx,epsfig}
%\usepackage{dcolumn}
%\usepackage{bm}
%\usepackage{ulem}

\usepackage{xspace} 

\usepackage{color}
\usepackage{xcolor}

\usepackage{lineno}
\linenumbers
\usepackage[export]{adjustbox}


\renewcommand{\baselinestretch}{1.2}

%\setlength{\textwidth}{17.5cm} 
%\setlength{\textheight}{24.cm}
%\setlength{\topmargin}{-1.5cm}
%\setlength{\oddsidemargin}{-0.4cm}


\newcommand{\be}{\begin{equation}}
\newcommand{\ee}{\end{equation}}
\newcommand{\bea}{\begin{eqnarray}}
\newcommand{\eea}{\end{eqnarray}}
\newcommand{\beaa}{\begin{eqnarray*}}
\newcommand{\eeaa}{\end{eqnarray*}}
\newcommand{\ba}{\begin{array}}	
\newcommand{\ea}{\end{array}}
\newcommand{\bi}{\begin{itemize}}
\newcommand{\ei}{\end{itemize}}
\newcommand{\ben}{\begin{enumerate}}
\newcommand{\een}{\end{enumerate}}

\newcommand{\bra}{\langle}
\newcommand{\ket}{\rangle}
\newcommand{\ra}{\rightarrow}
\newcommand{\lra}{\longrightarrow}
\newcommand{\overar}{\overrightarrow}
\newcommand{\wt}{\widetilde}
\newcommand{\td}{\tilde}


\newcommand{\lb}{\label}
\newcommand{\g}{\ensuremath{\gamma}\xspace}
\newcommand{\G}{\Gamma}
\newcommand{\e}{\epsilon}
\newcommand{\al}{\alpha}
\newcommand{\bt}{\beta}
\newcommand{\p}{\partial}
\newcommand{\dl}{\delta}
\newcommand{\Dl}{\Delta}
\newcommand{\ld}{\lambda}
\newcommand{\Ld}{\Lambda}
\newcommand{\vp}{\varphi}
\newcommand{\te}{\theta}
\newcommand{\Om}{\Omega}
\newcommand{\om}{\omega}
\newcommand{\sm}{\sigma}
\newcommand{\Sm}{\Sigma}

\newcommand{\A}{{\rm A}}
\newcommand{\B}{{\rm B}}
\newcommand{\U}{{\rm U}}
\newcommand{\F}{{\rm F}}
\newcommand{\SU}{{\rm SU}}
\newcommand{\Tr}{{\rm Tr}}
\newcommand{\Hom}{{\rm Hom}}

\newcommand{\FF}{{\mathsf F}}

\newcommand{\mcE}{{\mathcal{E}}}
\newcommand{\N}{{\mathcal{N}}}
\newcommand{\D}{{\mathcal{D}}}
\newcommand{\La}{{\mathcal{L}}}
\newcommand{\OO}{{\mathcal{O}}}
\newcommand{\M}{{\mathcal{M}}}


\newcommand{\mbE}{{\mathbb{E}}}
\newcommand{\Z}{{\mathbb{Z}}}
\newcommand{\R}{{\mathbb{R}}}
\newcommand{\C}{{\mathbb{C}}}
\newcommand{\NN}{{\mathbb{N}}}
\newcommand{\PP}{{\mathbb{C}}{\rm P}}

\newcommand{\HH}{{\mathcal{H}}}
\newcommand{\Hd}{{\mathcal{H}}^*}

\newcommand{\HI}{H~\textsc{i}\xspace}
\newcommand{\Htwo}{$\mathrm{H}_2$\xspace}
\newcommand{\hi}{$\mathrm{H\,\scriptstyle{I}}$\xspace}
\newcommand{\hii}{$\mathrm{H\,\scriptstyle{II}}$\xspace}
\newcommand{\hd}{$\mathrm{H}_2$\xspace}
\newcommand{\xco}{$X_\mathrm{CO}$\xspace}

\newcommand{\vx}{{\bf x}}

\newcommand{\Fermi}{\textit{Fermi}\xspace}
\newcommand{\LAT}{\textsl{LAT}\xspace}
\newcommand{\WMAP}{\textsl{WMAP}\xspace}
\newcommand{\Planck}{\textsl{Planck}\xspace}
\newcommand{\Suzaku}{\textsl{Suzaku}\xspace}

\newcommand{\SM}{Sample Model\xspace}

\newcommand{\sigmav}{\ensuremath{\langle \sigma v \rangle}\xspace}
\newcommand{\bbbar}{\ensuremath{b \bar b}\xspace}
\newcommand{\tautau}{\ensuremath{\tau^{+}\tau^{-}}\xspace}
\newcommand{\relic}{\ensuremath{2.2\times10^{-26}\cm^{3}\second^{-1}}\xspace}
\newcommand{\beff}{\ensuremath{b_{\rm eff}}\xspace}
\newcommand{\DM}{\ensuremath{\mathrm{DM}}}
\newcommand{\mDM}{\ensuremath{m_\DM}\xspace}


% local options
\newcommand{\onepic}{0.45}
\newcommand{\twocolumnwidth}{0.75}
%\newcommand{\twopic}{0.38}
\newcommand{\twopic}{0.3}
\newcommand{\twopicsp}{0.45}
\newcommand{\threepic}{0.25}
%\newcommand{\threepic}{0.18}
\newcommand{\fourpic}{0.28}
\newcommand{\twopicwca}{0.35}

\newcommand{\cmap}{_afmhot}


\newcommand{\red}{\textcolor{red}}
\newcommand{\blue}{\textcolor{blue}}
\definecolor{darkgreen}{rgb}{0.0, 0.7, 0.0}
\newcommand{\green}{\textcolor{darkgreen}}

\newcommand{\dima}[1]{\textcolor{blue}{(Dima: #1)}}
% local options

\begin{document} 


   %\title{Machine learning classification of unassociated \Fermi LAT sources}
   \title{Machine learning methods for probabilistic catalogs}

   %\subtitle{}

   \author{A. Bhat \thanks{\email{aakash.bhat@fau.de}}
          \inst{1}
          \and
          D. Malyshev \thanks{\email{dmitry.mayshev@fau.de}}
          \inst{1}
          }

   \institute{
             Erlangen Centre for Astroparticle Physics, Erwin-Rommel-Str. 1, Erlangen, Germany
             }

   \date{Received September 15, 1996; accepted March 16, 1997}

% \abstract{}{}{}{}{} 
% 5 {} token are mandatory
 
\abstract
% context heading (optional)
% {} leave it empty if necessary  
{
Classification of sources is one of the most important tasks in astronomy.
%It enables one to determine common features of the sources and understand better their nature.
Sources detected in one wavelength band, e.g., in gamma rays, may have several possible associations in other wavebands or
there may be no plausible association candidates.
}
% aims heading (mandatory)
{
In this work, we take unassociated sources in the third \Fermi-LAT point source catalog (3FGL) and suggest associations
to known classes of gamma-ray sources using machine learning methods trained on associated sources in the 3FGL.
}
% methods heading (mandatory)
{
We use several machine learning methods to separate \Fermi-LAT sources into two major classes: pulsars and active galactic nuclei (AGNs).
We evaluate the dependence of results on meta-parameters of the ML methods, such as the depth of the tree in tree-based classification methods and 
the number of layers in neural networks.
We test the performance of the methods with a test sample drawn from the associated sources in 3FGL.
We compare the predictions with the forth \Fermi-LAT catalog (4FGL).
}
% results heading (mandatory)
{
Summary of results
}
% conclusions heading (optional), leave it empty if necessary 
{}

\keywords{Methods: statistical --
                Catalogs
                %Galaxy: halo --
                %Galaxy: structure -- 
                %ISM: jets and outflows
               }

\maketitle
   
   
\tableofcontents


\section{Introduction}

Catalogs of gamma-ray point sources are typically designed to have low false detection rate. False detections may arise, for example, from statistical fluctuations of the background emission, from deficiencies in the diffuse emission model, or from an overlap of faint sources, which results in a detection of a single source.
The low false detection rate is achieved by setting a high statistical significance threshold, e.g., 4 or 5 sigma.
Although a high detection threshold helps to eliminate most of the false detections due to statistical fluctuations, it is not very effective against deficiencies of the background model or overlapping sources. 
Moreover, a high threshold removes many objects, which have a high chance to be point-like sources.
%In some problems including more sources with a higher contamination is beneficial, for example, in correlating gamma-ray PS with astrophysical neutrinos or in cross-correlation of the distribution of gamma-ray sources and large scale structures at different redshifts.
In other words, the catalogs are typically designed to be clean, but in some cases one may be interested to have a complete catalog. For example, one may want to have a list of all possible pulsar candidates among the unassociated sources in a catalog, which can be derived at the expense of many non-pulsars in the list.

The idea of probabilistic catalogs [Finkbeiner] is to include additional information, which describes a probability that a particular object is a point source or that a particular unassociated PS belongs to a certain class. 
%The probabilistic catalog can be implemented at the level of the PS detection, or it can also be implemented at the level of associations of already detected PS. 
For example, about one third of \Fermi-LAT sources have no firm associations with known Galactic or extragalactic sources. 
Although the associations are unknown, these sources can still be classified with some probabilities into, e.g., extragalactic or Galactic sources based on their position on the sky, properties of the gamma-ray flux and other features.
The classes can be further subdivided into various types of blazars or galaxies for extragalactic sources, or pulsars, pulsar wind nebulae, or supernova remnants for Galactic sources.
The classification probability is not unique, it depends on the classification method. The range of probabilities corresponding to different methods can serve as an estimate of the modeling uncertainty of the classification. In case of PS detection, one can derives probabilities that an object is a point source, a statistical fluctuation of the background, a deficiency of the background model, or an overlap of point sources. 
In this case, the probability will include not only the statistical probability but also the modeling uncertainties.

In this paper we will construct a probabilistic catalog using as an example classification of unassociated sources in the \Fermi-LAT catalogs. 
We will start with the Third \Fermi-LAT catalog (3FGL) and classify the unassociated sources into pulsars and AGNs using the associated sources in 3FGL for training of the classification algorithms.
We will use several machine learning algorithms for the classification, e.g., random forest, boosted decision trees, and neural nets
(since the number of features and the training sample are small, the neural networks will be rather shallow).
We will show applications of the probabilistic catalog for predicting the number of pulsars among the unassociated source and in construction of the source counts as a function of their flux, $dN/dS$.
Since unassociated sources on average have smaller flux than the associated ones, the $dN/dS$ distribution for the probabilistic catalog extends to lower fluxes relative to counting only the associated sources.
We will compare the prediction for the number of pulsars and the $dN/dS$ functions with the Forth \Fermi-LAT catalog (4FGL).




\section{Choice of methods}
\lb{sec:methods}

\subsection{General methodology}
We will use data-driven approach in constructing the probabilistic catalog.
The result depends on two inputs: data used for constructing the model and the choice of the method for classification.
The data in our case will be sources in 3FGL with known classes. 
We will split the data into training and testing subsets.
For the methods, we will consider four machine learning algorithms: boosted decision trees (BDTs),  random forests (RF),
logistic regression (LG), and neural networks (NN).
The resulting probabilities of classification depend on the choice of the classification algorithm.
Although some algorithms have slightly better performance on the test sample then others,
the overall performance is relatively similar.
As a result, we will report the classification probabilities for all four algorithms in the catalog, instead of selecting the 
``best'' one.
The difference among the predictions will serve as a measure of modeling uncertainty related to the choice of the classification algorithm.

\subsection{Discussion of the choice of the classification algorithms}
{\it Decision trees}
One of the most simple and transparent algorithms for classification is a decision tree.
In this algorithm, at each step the sample is split into two subsets using one of the input features.
The choice of the feature and the separating value are determined by minimizing an objective function, such as misclassification
error, Gini index, or cross-entropy.
This method is very intuitive, since at each step the results can be described in words, 
for example, at the first step, the sources can be split in "mostly" Galactic and extragalactic by a cut on the Galactic latitude.
At the next step, the high latitude sources can be further subsplit into millisecond pulsars and other sources, buy a cut on the spectral index around 1 GeV (pulsars have a hard spectrum below a few GeV) etc.
One problem with decision trees is overfitting: if the tree is too deep, then it will pick up particular cases of the training sample, while too shallow tree would not be able to describe the data well. As a result, one needs to be very careful in selecting the depth of the tree.
This problem can be avoided if a random subset of features is used to find a division at each node. This is the basis of the RF algorithm,
where the final classification is given by an average of several trees with random subsets of features used at each node.
Another problem with the simple trees is that it can miss the classification of some subsets of data. In BDT algorithms, the final classification is given by a collection of trees, where each new tree is created by increasing the weights of misclassified samples of the previous step. 
Finally, simple trees predict classes for the data samples, while we would like to have probabilities of classes (also known as soft classification).
RF and BDT algorithms, by virtue of averaging, provided probabilities. As a result, we will use RF and BDT algorithms rather than simple trees in this paper.

Tree-based algorithms, even after averaging in RF and BDT methods, have sharp edges among domains with different probabilities.
In LR algorithm, the probabilities of classes are by construction smooth functions of features.
In particular, for two-class classification the probability of class 0, given the set of features $x$, is modeled by sigmoid (logit) function
\be
\lb{eq:logit}
p_0(x) = \frac{1}{1 + e^{m(x)}}.
\ee
The probability of class 1 is then modeled as $1 - p_0(x)$.
If $m(x)$ is a linear function of features, then the boundary between the domains, defined, e.g., as $p_0(x) = 0.5$, will be linear.
More complicated boundaries can be modeled by taking non-linear functions $m(x)$.
Unknown parameters of the function $m(x)$ are determined by maximizing the log likelihood of the model given the known classes of the data in training sample.
A nice feature of the LR method is that it, by construction, provides probabilities of classes with smooth transitions among domains of different classes.
A limitation is that the form of the probability function is limited by the sigmoid function in Equation (\ref{eq:logit}).

We notice that if $m(x)$ is a linear function of features $x$, then the logistic regression model is obtained by an application of sigmoid function to a linear combination of input features.
This is in fact a single layer perceptron, or a neural network without hidden layers, with several input nodes (each node corresponds to
a features) and one output node, which corresponds to $p_0(x)$.
The output value is obtained by a non-linear transformation (sigmoid) of a linear combination of features.
Neural network with several hidden layers is obtained by a sequence of nonlinear transformations of linear combinations of features.
In particular, the values in the first hidden layer are obtained by a non-linear transformation of linear combinations of input features.
Then the values in second hidden layer are non-linear transformations of linear combinations of values in the first hidden layer etc.
In the context of neural networks, the non-linear transformations are called activation functions.
If the activation function for the output layer is sigmoid, then the output value (values) can be interpreted as probabilities.
We notice that in this case the neural network is can be expressed by a logistic regression for some function $m(x)$,
i.e., the neural network is then a particular way of constructing $m(x)$.
Thus the only difference between LR and NN for the classification problems is the construction of the function $m(x)$.
In this paper, for LR $m(x)$ will be constructed as a combination of low-order polynomials of the input features,
while for NN, $m(x)$ will be constructed by taking linear input features and several hidden layers, e.g., 4 or 5, 
in a fully connected neural network.

\subsection{Feature selection}

\subsection{Details of algorithms}





\subsection{Old}

Our methodology for classification was dependent on two things: The data that we had, which needed to be cleaned and the algorithms that we needed to apply. For this we decided on using the 3rd catalog of F-LAT (3FGL from hereon) for initial training and testing, the 4th catalog (FL8Y from hereon) for further testing and predictions, and machine learning algorithms like Random Forests, Logistic Regression, Decision Trees, and Neural Networks. All of the machine learning algorithms were taken from the python module sklearn, including Neural Networks. A neural network using Keras was also attempted; however, due to the classification being on only two classes, we discarded it in favour of the sklearn algorithm which was much faster.\\

Our data was similar to that used by Parkinson et. al. We cleaned the 3FGL catalog to have sources which were both associated and unassociated but with no missing values. We then used the associated sources which were classified as either AGNs (with multiplpe labels) or Pulsars, to get a list of 1905 sources. The rest of the sources without problematic values were then used as unassociated sources, which we used later on for testing and prediction. The FL8Y presented us with another way of testing the accuracy of our methods. We predicted the classifications of unasssociated sources in the 3FGL and used the FL8Y to check how many of these unassociated sources, which now had associations in the FL8Y, actually had the right prediction.\\

The raw data of the catalog had a lot of different features that could be used for classification. However, going by the previous studies, we decided on using the most important features, which included Flux density and the error on it, spectral index, the curvature, hardness ratios (as defined by Parkinson et. al.), variablity, and also the galactic latitude, the last of which was used even in the classification of AGN and Pulsars (as opposed to Parkinson, who used it only for the young and milli-second pulsar distinction). In features where the values were high, we used the logarithmic scale to better seperate the sources. The complete list of sources, along with some statistics, is given in the table below. The influence of the features on the classification, especially the differences in the various methodologies is discussed in much more detail in the next section.\\

One of the main aims of our project was to understand and optimize the machine learning methods which we were using. So apart from the features which were in the data itself, we also theorized and experimented with the parameters of the algorithms themselves. We wanted to find the fastest and cost-effective way of using certain methods, without going into regimes of under and over-fitting the data. Parameters which we studied range from Depth and Number of trees in Forest based methods to the number of hidden layers and epochs in neural networks. The details are given in the next section, where we discuss our expectations and the resulting behaviour of our algorithms.\\
  
In our general the Methodology was as follows.
\ben
\item
Split the PS with known classification into learning and test samples.
\item
Use the learning sample for training and for selection of features.
In particular, continuous parameters, such as the thresholds in the decision trees or mixing matrices in neural networks, are determined from the learning sample. 
\item
Meta-parameters, which encode the complexity of the methods, such as the depth of the decision trees,
are determined from the best performance on the test sample.
\een

After the above had been completed we were ready both with our final data and our optimum algorithms. We then applied and sought the results using both the catalogs in our possession. This is discussed in detail in section 4 and 5.\\


When applied on the 3FGL known sources, using 1500 sources to train and the rest to test on, we found (for 10 seeds) the following: \\

\begin{table}[!h]
    \tiny
    \centering
    \renewcommand{\tabcolsep}{1mm}
\renewcommand{\arraystretch}{1.5}

    \begin{tabular}{|c|c|c|}
    \hline
    Algorithm Name&Parameters & Accuracy\\
    \hline
    Random Forest& 50 trees and 12 max depth & 97.91        \\
    \hline
    Neural Network & 200 epochs and 20 neurons in 1 layers       &  98.2 \\
    \hline %\midrule   -> aakash do you mean this?
    Gradient Boost& 50,15      &   96.78  \\
    \hline %\midrule   -> aakash do you mean this?
    Logistic Regression& all solvers &<94  \\
    \hline
     
    \end{tabular}

    \caption{Testing Accuracy of 4 algorithms on 3FGL data}
    \label{tab:my_labe2l}
\end{table}

\subsection{Details of the analysis}

\subsection{Data and Features}

The total number of sources, including unassociated and associated, in the two catalogs is shown below. \\
\begin{figure}[h]
%\centering
\includegraphics[width=\onepic\textwidth]{plots/correlation.pdf}
\caption{Correlation matrix for the most important features}
\label{fig:corr}
\end{figure}
[Add Table]\\

The features used for our analysis follow the same idea as the previous studies. The features, along with statistical and methodological details, are given below. A correlation matrix is presented for the most important features as well. The matrix is important for the case where there might be redundant features, in which case using only one of the two features would be a better idea.\\



[Add Table of features for both catalogs]\\

Our initial hypothesis was that certain features would be more important for classification than others. For instance, as shown below, one can see a clear distinction between the regimes of AGNs and Pulsars, based on spectral idex and significant curvature. [Add image] While not clearly obvious from the get go, we were also interested in comparing the importance of features based on the algorithms that we were using. Due to the difference in the basic method of Random Forests and Neural Networks, we expected a slight shift in their reliance on certain features. Despite that we hypothesized that features with the most contribution would be among spectral index, variability, and the curvature; as already observed by Parkinson et. al.\\

\begin{figure}[h]
%\centering
\includegraphics[width=\onepic\textwidth]{plots/signifcurvvsspecind.pdf}
\caption{Differences in AGNs and PSRs from the 3FGL catalog}
\label{fig:corr}
\end{figure}


\subsection{}


\ben
\item
Describe the features that we use for the analysis.
\item
Describe the objective function for minimization (accuracy of classification on learning sample).
Weighted objective function: give more weight to pulsars, since there are fewer of them in the catalog.
\item
Learning curve using all features?
{\it Plot: classification accuracy using the total list of features for learning and test sample as a function of complexity parameter.}
\item
Selection of the most important features.
{\it Table: features vs algorithms. Columns: algorithms, rows: features, values: significance.}
\item
Selection of meta-parameters.
{\it Plot: classification results for the test sample using a subset of features.}
\item
Train the final classifier.
{\it Table: classification accuracy of the final classifiers for different algorithms using the test sample from 3FGL.}
\een

Discuss the general features of the optimal algorithms: which features turn out to be important, what is the depth of the trees, the number of trees in random forests, the depth and number of internal nodes in the neural networks. \\

\begin{figure}[h]
%\centering
\includegraphics[width=\onepic\textwidth]{plots/Rf_maxdepth_oobscore_glat}
\caption{
Example of a figure for one column.
}
\label{fig:Maps_data}
\end{figure}


\begin{figure*}[h]
%\centering
\includegraphics[width=\twopicsp\textwidth]{plots/Rf_maxdepth_oobscore_glat}
\includegraphics[width=\twopicsp\textwidth]{plots/Rf_maxdepth_oobscore_glat}
\caption{
Example of a figure for both columns.
}
\label{fig:Maps_data}
\end{figure*}

Our hypothesis about feature importances turned out to be correct, as curvature, variability, and spectral index were the most important features. The last hardness ratio was also seen to be quite important, most probably reflecting the end of the spectrum where the AGNs and PSRs shift from each other. These values are given in the table below. \\

\begin{table}[!h]
    \tiny
    \centering
    \renewcommand{\tabcolsep}{1mm}
\renewcommand{\arraystretch}{1}

    \begin{tabular}{|c|c|c|}
    \hline
    Feature Name&  RF (50,15)& GB (50,15)\\
    \hline
    Flux Density& 0 & 0        \\
    \hline
    Unc Energy Flux100& 0     & 0 \\
    \hline %\midrule   -> aakash do you mean this?
   Spectral Index & 0.16      &   0.07  \\
    \hline %\midrule   -> aakash do you mean this?
    Significant curvature& 0.28 &0.47  \\
    \hline
   var&  0.11    &  0.21  \\
    \hline %\midrule   -> aakash do you mean this?
    hr12& 0.06 &0.04 \\
    \hline
     hr23& 0.04 &0.02 \\
    \hline
    hr34& 0.06 &0.04 \\
    \hline
   hr45& 0.22 &0.10 \\
    \hline
    GLAT&0.04&0.01\\
    \hline
    \end{tabular}

    \caption{Feature importances for different Algorithm}
    \label{tab:feat_imp}
\end{table}

These importances were found to be consistent for various different algorithm parameters. So while the value might change a bit for different tree architechtures, for instance, the importances of these features were still pronounced. \\
\subsection{Comparison of the classification algorithms}

{\it Plot: classification domains for a pair of features (or different pairs of features, e.g., latitude vs index, index vs curvature, latitude vs variability).}

Probabilistic classification? Result: probability for a source to belong to a particular class.
Result of classification: table of sources with probabilities for different algorithms.
Final probability: the probability for one of the algorithms (for the most precise one?) and uncertainties determined from the other algorithms.

Discuss a few examples where algorithms give different predictions (are these sources at the boundaries of the domains).

Discuss examples where algorithms misclassify sources from the test sample.\\


In the case of test data, we worked with three different classification algortithms, namely Random Forests, Ada Boost, and Neural Networks. Here we were mostly concerned with tweaking the parameters of the classification algorithms involved, minimizing the cost of computation and aiming for the most efficient way of classification. \\

\subsubsection{Random Forests}


\begin{figure}[h]
%\centering
\includegraphics[width=\twopicsp\textwidth]{plots/depthvsscore_rf_10seeds_20trees.pdf} \\
\includegraphics[width=\twopicsp\textwidth]{plots/treesvsscore_RF_10seeds_10maxdepth}
\caption{
Random Forests on Testing Data
}
\label{fig:Maps_data}
\end{figure}

The two main parameters involved in Random Forests are the number of trees and the maximum depth of the trees involved. Figures below shows one instance of the accuracy as a function of maximum depth when the number of trees was kept constant, and as a function of number of trees when the maximum depth was kept constant. \\


\subsubsection{Neural Networks}

In the case of neural networks we were concerned with the number of epochs that one would need to tweak, along with a dependence on the number of neurons in the hidden layers. A final improvement involved checking whether multiple hidden layers would actually add to such a classification algorithm or not.\\
As can be seen in the figure, a complex network with two hidden layers (100 and 5 neurons) reaches the maximum accuracy pretty fast. The results becoming more consistent at higher epochs. A similar result was found for a network having two hidden layers but with only 20 neurons in the first layer. However, such networks could also lead to overtraining, and therefore it is important to check whether such a high accuracy could perhaps be reached by less complicated algorithms, which would drastically reduce the chances of overtraining and allow for a more flexible classificaition methodology.\\

\begin{figure}[h]
%\centering
\includegraphics[width=\onepic\textwidth]{plots/epochsvsscore_10seeds_100_2}
\caption{
Example of a figure for one column.
}
\label{fig:Maps_data}
\end{figure}

\begin{figure}[h]
%\centering
\includegraphics[width=\onepic\textwidth]{plots/epochsvsscore3_10seeds.pdf}
\caption{
Neural Network
}
\label{fig:Maps_data}
\end{figure}

A consistent and accurate result is found even for networks with only one hidden layer with 20 and 5 neurons in the hidden layer respectively. There seems to be no significant dependence for the number of epochs above 30, and even a simple network with one layer and 5 neurons shows a high accuracy for only 30-40 epochs.\\


\begin{figure*}[h]
%\centerin
\includegraphics[width=\twopicsp\textwidth]{plots/epochsvsscore1_10seeds.pdf}
\includegraphics[width=\twopicsp\textwidth]{plots/epochsvsscore2_10seeds.pdf}
\caption{
Neural networks
}
\label{fig:Maps_data}
\end{figure*}

\begin{figure}[h]
%\centering
\includegraphics[width=\onepic\textwidth]{plots/epochsvsscore1_10seeds_1layer_5.pdf}
\caption{
Neural Network
}
\label{fig:Maps_data}
\end{figure}	




\section{Prediction for unassociated sources in 3FGL and comparison with 4FGL}
\lb{sec:3FGLprediction}



{\it Plot: add unassociated sources on the plots with domains for the best algorithm.}

Comment: Create a table with sources which are more likely to be pulsars (select about 20 the most likely candidates).
Compare the accuracy of the algorithm for the sources which have an associate now in the 4FGL catalog.\\

In this section we use the best algorithms from the previous section to predict classes for the unassociated sources in the 3FGL. We then use the associations which exist for some of these sources in the 4FGL to check the accuracy of our methods on the unassociated data. In this section we work only with Random Forests, Neural Networks, AdaBoost, and Logistic Regression. \\


\subsection{3FGL Unassociated sources with Association in 4FGL}
There were a total of 286 sources without associations in 3FGL but which had a corresponding assocation in 4FGL. We trained our algorithms on the entire associated data from the 3FGL, and then used tested our algorithms on these 286 sources. The probabilistic version is discussed in the next section.  \\

The following were the optimized results we obtained for this case:\\

\begin{table}[!h]
    \tiny
    \centering
    \renewcommand{\tabcolsep}{1mm}
\renewcommand{\arraystretch}{1.5}

    \begin{tabular}{|c|c|c|}
    \hline
    Algorithm Name&Parameters & Accuracy\\
    \hline
    Random Forest& 50 trees and 12 max depth & 96.22   \\
    \hline
    Neural Network & 200 epochs, 20 neurons, tanh     &  95.97 \\
    \hline %\midrule   -> aakash do you mean this?
    Gradient Boost& 20,5    &   95.8  \\
    \hline %\midrule   -> aakash do you mean this?
    Logistic Regression& all solvers &to put  \\
    \hline
     
    \end{tabular}

    \caption{Testing Accuracy of 4 algorithms on 3FGL unassociated data}
    \label{tab:my_labe2l}
\end{table}


As can be seen above, the best accuracies were found with less complicated models, which allowed bias to be low. The models were complicated enough to neither under, nor overtrain. \\

\subsection{3FGL Probabilistic classification} 


\section{Application of probabilistic catalogs for population studies}
\lb{sec:dNdS}



\begin{figure}[h]
%\hspace*{-1cm}
\includegraphics[width=0.4\textwidth]{plots/logN_logS_AGN.pdf}
%\hspace*{-1cm}
\includegraphics[width=0.4\textwidth]{plots/logN_logS_PSR.pdf}
\caption{Log N - log S.}  
\label{fig:logN_logS}
\end{figure}




\begin{figure}[h]
%\hspace*{-1cm}
\includegraphics[width=0.4\textwidth]{plots/logN_logS_diff_AGN.pdf}
%\hspace*{-1cm}
\includegraphics[width=0.4\textwidth]{plots/logN_logS_diff_PSR.pdf}
\caption{Fraction of unassociated sources relative to associated ones.}  
\label{fig:unass_vs_ass_frac}
\end{figure}



In this Section we show how probabilistic catalogs can be used, for instance, for population studies,
which can be used, for instance, to determine the contribution of unresolved point sources to isotropic gamma-ray emission.
Understanding the origin of the isotropic gamma-ray emission is important for placing strong constraints
on dark matter annihilation into gamma rays or evaporation of primordial black holes
as well as constraining the origin of astrophysical high energy neutrino flux.


\section{Conclusions}
\lb{sec:conclusions}

In the paper we determine the probabilities of classification of unassociated sources in the 3FGL and 4FGL \Fermi-LAT catalogs.
The probabilities are calculated with 4 different ML methods: random forests, boosted decision trees, logistic regression and neural networks.
The algorithms were trained and tested with associated sources.
We have also scanned some meta-parameters of the algorithms, such as depth of the trees, the number of trees, the number of neurons etc. in order to determine optimal parameters which do not create overfitting of data and provide good accuracy of classification.
The accuracy, which we obtained for the 3FGL catalog for all four algorithms, was about 97\%.
We have also checked the accuracy of classification by selecting unassociated sources in 3FGL, which have associations in 4FGL.
If we take the 4FGL associations as the true value, then the accuracy of classification in this subset of sources is between 93\% and 95\%.
As one can see from Figure \ref{fig:3FGL_vs_4FGL_classes}, the misclassified sources have spectral parameters in 3FGL which are typical of the other class, i.e., the misclassification can be due to problems with reconstructing the spectrum of the sources.

As a result of the analysis we create catalogs with probabilistic classifications of sources, where for each source and each class (i.e., AGNs and pulsars in our case) we report class probabilities for each of the four algorithms, i.e., 8 columns labeled by classes and by methods: ``AGN\_BDT'', ``PSR\_NN'' etc.
We report the classification probabilities not only for the unassociated sources, but also for the associated ones, which can be used to find outliers.
An advantage of such probabilistic classification is that a threshold on probability for selecting, e.g., pulsar candidates, can be chosen by the user based on his or her needs.
For example, in a search of new pulsars, one can select a low threshold in order to avoid missing possible pulsars.
In a derivation of an average property of the class, e.g., spectral index or cutoff energy, one can select a high threshold in order to avoid contamination from the other class, in addition one can use weighting by probability.
However, one should be careful here, since some of the properties of sources were used for the determination of probabilities,
the average of these features weighted by the probabilities can be biased compared to an average derived from multiwavelength associations.

As an example of the application of the probabilistic catalogs, we derive the expected number of sources in the catalog as a function of their flux, including the unassociated sources.
As a consistency check, we compare the counts of associated sources to the sums of probabilities for associated sources.
We find that correcting for the contribution of sources other than AGNs and pulsars plays an important role for estimation of the expected number of sources in a particular class.
We find the total expected number of AGNs and pulsars in 3FGL and 4FGL catalogs by adding the class probabilities in the unassociated sources to the source counts of associated sources and correcting for the contribution of other classes in the unassociated sources.
In particular, we find that the total expected number of pulsars is about two times larger than the number of associated pulsars.



\newpage
\bibliography{ML_3FGL_papers}  

\begin{appendix}
\section{Tests of additional meta-parameters}
\lb{sec:app}

In this appendix we discuss tests of some meta-parameters, which had a relatively little effect on the 
accuracy of the algorithms. For these tests we use the 2-class classification in the 3FGL catalog.

LR algorithm has two meta-parameters: regularization and tolerance. 
%These features are the limit up to which one wants to go before accepting a solution. 
The effect of the choice of these parameters on accuracy is less than 1\% (Figure \ref{fig:LR_tol_reg}). 
Therefore we used the default values for these parameters (tolerance is 1$e^{-4}$ and regularization parameter is set at 1). The same was used in the 3-class case where the change from GLON to cos(GLON) was seen as the main improvement for LR algorithm.
%\dima{Did we change the regularization in the oversampling 3-class case? I guess it was probably not as important as the change from GLON to cos(GLON).}. 
\begin{figure}[h]
\centering
\includegraphics[width=\twopicsp\textwidth]{plots/lr_train_reg.pdf}
%\includegraphics[width=\twopicsp\textwidth]{plots/unassoc_complex.pdf}
\caption{Dependence of LR on tolerance and regularization. 
}
\label{fig:LR_tol_reg}
\end{figure}

In Figure \ref{fig:nn_nn} we show the effect of adding the second hidden layer in the NN algorithm.
The difference between the best accuracies with the additional hidden layer is less then 1\%
compared with the NN with one hidden layer (cf. Table \ref{tab:selected_algs}).
\begin{figure}[h]
\centering
\includegraphics[width=\twopicsp\textwidth]{plots/nn_2layers_3fgl.pdf}
%\includegraphics[width=\twopicsp\textwidth]{plots/unassoc_complex.pdf}
\caption{Dependence of NN on the number of neurons in the second hidden layer, for 11 neurons in the first hidden layer.
}
\label{fig:nn_nn}
\end{figure}

We summarize features and their statistics,
which we use for probabilistic classification of sources in the 3FGL and 4FGL-DR2 catalogs,
in Tables \ref{tab:3FGL_features} and \ref{tab:4FGL_features} respectively. 
We show the feature importances for the 2-class classification of 4FGL-DR2 sources in Table \ref{tab:feat_imp2}.


\pgfplotstableread[col sep=comma]{tables/features/3fglassocfeaturesAGNPSRnewfeats.csv}\tablea
\begin{table}
\centering
\resizebox{0.45\textwidth}{!}{
\pgfplotstabletypeset[
columns={Name,Mean,SD,Minimum,Maximum},
column type=c,
string type,
every head row/.style={before row=\hline\hline,after row=\hline,},
every last row/.append style={after row={\hline} },
every first column/.style={column type/.add={}{}},
every last column/.style={column type/.add={}{}},
columns/Name/.style={column name=Feature Name,string replace*={_}{\textunderscore}},
columns/Mean/.style={column name=Mean,column type=c,numeric type,fixed,precision=2},
columns/SD/.style={column name=Standard Deviation,numeric type,fixed,precision=2},
columns/Minimum/.style={column name=Minimum,numeric type,fixed,precision=2},
columns/Maximum/.style={column name=Maximum,numeric type,fixed,precision=2},
skip rows between index={11}{25}
]{\tablea}
}
\vspace{0.2cm}
\caption{Statistics of features used for 2 class probabilistic classification of the 3FGL sources.
\lb{tab:3FGL_features}}
\end{table}

\pgfplotstableread[col sep=comma]{tables/features/4fgldr2agnpsrfeatures.csv}\tableaf
\begin{table}
\centering
\resizebox{0.45\textwidth}{!}{
\pgfplotstabletypeset[
columns={Name,Mean,SD,Minimum,Maximum},
column type=c,
string type,
every head row/.style={before row=\hline\hline,after row=\hline,},
every last row/.append style={after row={\hline} },
every first column/.style={column type/.add={}{}},
every last column/.style={column type/.add={}{}},
columns/Name/.style={column name=Feature Name,string replace*={_}{\textunderscore}},
columns/Mean/.style={column name=Mean,column type=c,numeric type,fixed,precision=2},
columns/SD/.style={column name=Standard Deviation,numeric type,fixed,precision=2},
columns/Minimum/.style={column name=Minimum,numeric type,fixed,precision=2},
columns/Maximum/.style={column name=Maximum,numeric type,fixed,precision=2},
%skip rows between index={17}{28}
]{\tableaf}
}
\vspace{0.2cm}
\caption{Statistics of features used for 2 class probabilistic classification of the 4FGL-DR2 sources.
\lb{tab:4FGL_features}}
\end{table}


\begin{table}[!h]
\tiny
\centering
\renewcommand{\tabcolsep}{1mm}
\renewcommand{\arraystretch}{1}

\begin{tabular}{c c c}
\hline
\hline
Feature & RF: 50, 6& BDT:100, 2\\
\hline
{ $\ln$(LP\_SigCurv)}&  0.297  & 0.465   \\
{LP\_beta}&0.151&0.109\\
{ $\ln$(Variability\_Index)} &0.085& 0.253   \\
$\ln$(Unc\_Energy\_Flux100)& 0.081&0.059  \\
$\ln$(Energy\_Flux100) & 0.076&0.008   \\
HR56&0.071& 0.015  \\
Unc\_LP\_Index & 0.067&0.009  \\
HR34& 0.035&0.005  \\
$\ln$(Pivot\_Energy)&0.031&0.006\\
HR23 &0.025& 0.005     \\
 LP\_Index& 0.015&0.016  \\
HR67&0.015&0.010\\
HR45&0.015&0.006\\
GLON&0.013&0.017\\
HR12&0.009&0.004\\
GLAT&0.007&0.003\\
\hline
\end{tabular}
\vspace{0.4cm}
\caption{Feature importances for RF and BDT algorithms ordered by decreasing importance 
in the case of RF algorithm for 4FGL-DR2.
}
\label{tab:feat_imp2}
\end{table}



\end{appendix}

\end{document}

