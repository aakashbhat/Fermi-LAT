\section{Introduction}

Machine learning algorithms have been around for some time. Their use in classification, etc. is well studied. However, it is only recently that machine learning has found it's way to astronomical classifications and tasks. Their use, especially with the growth of neural networks has increased exponentially in the past few years. They are now being used in classification of astrophysical sources, as well as in other areas such as reconsttruction of particle tracks (for instance in Icecube etc.). \\
The fermi large area telescope (LAT) was launched in 2008 for the detection of photons in the 	gamma-ray regime. The LAT team has since then released 4 catalogs with the 8-year list being released in 2019. These catalogs provide a list of sources, which include AGNs and Pulsars. While a lot of them are associated, many of the sources still remain unassociated.\\
Attempts to classify these unassociated sources have previously been performed. In 2016, Parkinson et. al. used statistical and machine learning methods like Random Forests and logistic regression to try and classify sources in the 3rd catalog released by the LAT team. They trained on 70 \% of the associated sources in the catalog and then tested their results on the rest of the 30 \% sources. The methods were used to classify AGNS and Pulsars (and seperately young and millisecond pulsars) and showed accuracy of up to 97\%.\\
In our paper we present our machine learning algorithms and go deeper into their working and data analysis strategies.\\
\lb{sec:intro}



