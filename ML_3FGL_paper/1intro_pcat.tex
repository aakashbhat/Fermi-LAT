\section{Introduction}

Multiwavelength association of astronomical sources is important for understanding their nature.
Unfortunately, in many cases a firm association of sources at different wavelength is not possible.
For example, about one third of gamma-ray sources in Fermi-LAT catalogs are unassociated
\citep{2010ApJS..188..405A, 2012ApJS..199...31N, 2015ApJS..218...23A, 2020ApJS..247...33A}.
It is useful to know at least to which class the unassociated sources belong to or, which is more typical,
what are the probabilities for the source to belong to different classes.
In this paper we use different machine learning algorithms to find probabilistic classification of
unassociated sources in \Fermi-LAT third (3FGL) and fourth (4FGL) catalogs \citep{2015ApJS..218...23A, 2020ApJS..247...33A}.
We will refer to the catalogs, where the classification of the sources is given by probabilities as probabilistic catalogs.
In general, the classes may include the possibility that a source is not a real source but a fluctuation of a background 
or that a source is an overlay of two sources etc.
All these possibilities can be included in the probabilistic catalogs, which were previously introduced for optical sources 
\citep[e.g.,][]{2010EAS....45..351H, 2013AJ....146....7B}
and for gamma-ray sources \citep{2017ApJ...839....4D}.
Bayesian association probabilities were also introduced in the 4FGL catalog \citep{2020ApJS..247...33A} for faint sources.
%Previous classification of unassociated \Fermi-LAT sources using machine learning methods include 
The goal of this paper is to construct probabilistic classification of sources in the 3FGL and 4FGL catalogs using ML algorithms.
Previously probabilistic classification of unassociated 3FGL sources was performed by \cite{2016ApJ...820....8S}.
In this work, we add neural networks for classification of 3FGL sources, compare results of classification
with associations in 4FGL and perform probabilistic classification of unassociated 4FGL sources.

Catalogs of gamma-ray point sources are typically designed to have low false detection rate. 
Nevertheless, 469 sources out of 3033 in the 3FGL catalog \citep{2015ApJS..218...23A} have no counterparts 
in the 4FGL catalog \citep{2020ApJS..247...33A}.
This is much larger than the expected false detection rate in 3FGL arising from statistical fluctuations.
For the majority of sources in 3FGL without counterparts in 4FGL the problem is not the false detection, 
but rather the association.
For example, some sources can be detected due to deficiencies in the Galactic diffuse emission model.
In this case, the statistical significance of the detection is high, but the association is wrong: the sources should be classified as
a part of the Galactic diffuse emission rather than point-like sources.
Another reason could be that two (or more) point-like sources in 3FGL are associated to a single extended source in 4FGL,
or a single source is resolved into two sources.
Again, this is a problem of classification (or association) rather than false detection.

Another reason for an absence of a previously detected source in a new catalog is variability.
In particular, flat spectrum radio quasars (FSRQs) are highly variable active galactic nuclei (AGNs).
If a source was active during the observation time of 3FGL but inactive afterwards, 
then its significance in the 4FGL can be below the detection threshold.
The problem here is connected to a selection of a hard detection threshold of $TS = 25$ for 3FGL and 4FGL catalogs.
Selection of a lower detection threshold could help to keep the variable sources inside the catalog, 
but it will not solve the problem, since the variable sources near the lower threshold can also disappear in the new catalog.
Moreover, lower threshold would lead to more false detections due to fluctuations of the background.
Thus, on the one hand, lower threshold can be useful in studies, where a more complete list of sources is desirable,
while the higher false detection rate is admissible. On the other hand, lower threshold can be problematic for studies where 
a clean sample is necessary. 
The problem of the detection threshold selection can be ameliorated with the development of a probabilistic catalog.
In this catalog, each point-like object detected above a certain relatively low confidence level
are probabilistically classified into classes, which include the statistical fluctuations class.
At low confidence, the probability for a source to be due to a fluctuation is high.
This probability is decreasing as the significance of sources increases.
Apart from the statistical fluctuations class, classes can include various types of Galactic and extra-galactic sources, diffuse emission deficiencies, extended sources.
%Even the statistical fluctuation of the background can be viewed as one of the classes: in this case, objects with small statistical significance will have a high probability to be associated to the statistical fluctuation class.
A user of such a catalog will have the freedom to choose the probability threshold for the class that he or she is interested in.








\begin{comment}
False detections may arise, for example, from statistical fluctuations of the background emission, from deficiencies in the diffuse emission model, or from an overlap of faint sources, which results in a detection of a single source.
The low false detection rate is achieved by setting a high statistical significance threshold, e.g., 4 or 5 sigma.
Although a high detection threshold helps to eliminate most of the false detections due to statistical fluctuations, it is not very effective against deficiencies of the background model or overlapping sources. 
Moreover, a high threshold removes many objects, which have a high chance to be point-like sources.
%In some problems including more sources with a higher contamination is beneficial, for example, in correlating gamma-ray PS with astrophysical neutrinos or in cross-correlation of the distribution of gamma-ray sources and large scale structures at different redshifts.
In other words, the catalogs are typically designed to be clean, but in some cases one may be interested to have a complete catalog. For example, one may want to have a list of all possible pulsar candidates among the unassociated sources in a catalog, which can be derived at the expense of many non-pulsars in the list.


The idea of probabilistic catalogs [Finkbeiner] is to include additional information, which describes a probability that a particular object is a point source or that a particular unassociated PS belongs to a certain class. 
%The probabilistic catalog can be implemented at the level of the PS detection, or it can also be implemented at the level of associations of already detected PS. 
For example, about one third of \Fermi-LAT sources have no firm associations with known Galactic or extragalactic sources. 
Although the associations are unknown, these sources can still be classified with some probabilities into, e.g., extragalactic or Galactic sources based on their position on the sky, properties of the gamma-ray flux and other features.
The classes can be further subdivided into various types of blazars or galaxies for extragalactic sources, or pulsars, pulsar wind nebulae, or supernova remnants for Galactic sources.
The classification probability is not unique, it depends on the classification method. The range of probabilities corresponding to different methods can serve as an estimate of the modeling uncertainty of the classification. In case of PS detection, one can derives probabilities that an object is a point source, a statistical fluctuation of the background, a deficiency of the background model, or an overlap of point sources. 
In this case, the probability will include not only the statistical probability but also the modeling uncertainties.

\end{comment}



In this paper we construct a probabilistic catalog using, as an example, classification of unassociated sources in \Fermi-LAT catalogs. 
We start with the 3FGL catalog and classify the unassociated sources into pulsars and AGNs using the associated sources in 3FGL for training of the classification algorithms.
We use several machine learning algorithms for the classification, e.g., random forest, boosted decision trees, logistic regression,
and neural networks.
We show applications of the probabilistic catalog for predicting the number of pulsars and AGNs among the unassociated source and in construction of the source counts as a function of their flux, $N(S)$.
Since unassociated sources on average have smaller flux than the associated ones, 
we expect that the $N(S)$ distribution for unassociated sources has a more important contribution to the total expected number of sources
for smaller fluxes compared to counting only the associated sources.
We compare the predictions for the $N(S)$ functions based on the 3FGL catalog with the 4FGL catalog.
We also probabilistically classify unassociated sources in the 4FGL catalog.


