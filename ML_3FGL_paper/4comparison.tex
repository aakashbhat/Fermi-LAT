\section{Application of probabilistic catalogs for population studies}
\lb{sec:pop_studies}

\subsection{Number of sources as a function of flux}
\lb{sec:dNdS}



\begin{figure*}[h]
\center
%\hspace*{-1cm}
\includegraphics[width=0.45\textwidth]{plots/N_logS_3FGL_PSR_SazP_add_os.pdf}
%\hspace*{-1cm}
\includegraphics[width=0.45\textwidth]{plots/N_logS_4FGL_PSR_add_os.pdf} \\
\includegraphics[width=0.45\textwidth]{plots/N_logS_3FGL_AGN_SazP_add_os.pdf}
%\hspace*{-1cm}
\includegraphics[width=0.45\textwidth]{plots/N_logS_4FGL_AGN_add_os.pdf}
\caption{Cumulative number of sources as a function of their flux.}  
\label{fig:logN_logS}
\end{figure*}


\begin{figure}[h]
\center
%\hspace*{-1cm}
\includegraphics[width=0.45\textwidth]{plots/N_logS_diff_AGN.pdf}
%\hspace*{-1cm}
\includegraphics[width=0.45\textwidth]{plots/N_logS_diff_PSR.pdf}
\caption{Fraction of unassociated sources relative to associated ones.}  
\label{fig:unass_vs_ass_frac}
\end{figure}



In this section we show how probabilistic catalogs can be used, for instance, for population studies.
One of the most important questions in gamma-ray astronomy is contribution of point sources, 
e.g., AGNs, to the extragalactic gamma-ray flux \citep[e.g.,][]{2010ApJ...720..435A, 2011ApJ...738..181M, 2016PhRvL.116o1105A, 2016ApJS..225...18Z, 2016ApJ...826L..31Z, 2016ApJ...832..117L, 2018ApJ...856..106D}:
if most of the extra-galactic emission is explained by point sources, then one can put stringent constraints, 
e.g., on  dark matter annihilation or decay into gamma rays 
\citep{2015ApJ...800L..27A, 2015PhRvD..91l3001D, 2015JCAP...09..008F, 2015PhR...598....1F, 2017ChPhC..41d5104L} or 
on evaporation of primordial black holes \citep{2010PhRvD..81j4019C}.
%and on the origin of astrophysical high energy neutrino flux  \citep{2013PhRvD..88l1301M, 2017ApJ...836...47B}.
In particular, it is important to understand the contribution to the population of AGNs from the unassociated sources.
A probabilistic catalog provides an answer to the question: how many sources among the unassociated ones are expected to belong to different classes, such as pulsars or AGNs. 
One can calculate the total expected number of AGNs or pulsars among the unassociated sources, or calculate the contribution as a function of one or more parameters.
In this section we determine the numbers of AGNs and pulsars as a function of their flux.

In Figure \ref{fig:logN_logS} we show the cumulative number of AGNs and pulsars with flux above 1 GeV larger than the
value on the x-axis.
Solid lines show the actual counts of sources (AGNs or pulsars) in the 3FGL and 4FGL catalogs.
As a consistency check of the method, we calculate the AGN- and PSR-like probabilities for associated sources.
The sum of probabilities (uncorrected for sources other than AGNs and pulsars) for LR algorithm are shown by dotted purple lines.
In order to correct the expected number of AGNs among associated sources for AGN-like probabilities in ``other'' sources, 
we subtract the corresponding AGN-like probabilities in each flux band:

\be
\lb{eq:assoc_ev}
N_{\rm AGN}^{\rm ass}  = \sum_{i \in \rm ass} p^i_{\rm AGN}\,\, - \sum_{i \in \rm ass\,other} p^i_{\rm AGN}.
\ee
The corrected sums of probabilities for LR method are shown by dashed purple lines.
The purple bands show the envelope of the sum of corrected probabilities for the four methods used in this paper.
We see that the counts of associated sources, AGNs and pulsars, are consistent with the expected number of associated sources
calculated from the class probabilities of associated sources.
This conclusion is not very surprising since we used associated sources for training of ML algorithms.
It is important to note that correction for ``other'' sources is important for consistency of the sum of probabilities and the number of associated sources.
We have also compared the sums of probabilities for the 3FGL associated sources in \cite{2016ApJ...820....8S}.%
%\footnote{The data is downloaded from \url{https://www.physics.hku.hk/~pablo/pulsarness.html}.}
% !!! resolve the footnote issue?
The sum of probabilities for associated sources in the LR case uncorrected for ``other'' sources are shown by dotted black line,
while the sums corrected for ``other'' sources are shown by black dash-dotted lines.
The gray band is the envelope of the two methods (LR and RF) used by \cite{2016ApJ...820....8S}.
We see that the sum of probabilities for pulsars overpredicts the pulsar counts in 3FGL, correction for ``other'' sources makes the prediction 
more consistent with the counts of pulsars, although it is still overpredicting them.

We note that the expected number of associated pulsars in 3FGL and 4FGL catalogs (purple band) is slightly below the 
actual counts of pulsars, this can be due to the fact that the number of AGNs in the training sample is much larger than the number of pulsars.
\cite{2016ApJ...820....8S} have used weights in the training sample inversely proportional to the number of elements in the class,
i.e., a pulsar has more weight in training than an AGN.
This can be a reason for an overprediction of the number of associated pulsars, compared to the actual count.


The predictions for the number of AGNs and pulsars among the unassociated sources corrected for ``other'' sources 
added to the 3FGL and 4FGL source counts are shown by green dashed lines (for the LR case).
The green bands show the corresponding envelopes for the four ML algorithms.
We assume that the fractional contribution of other sources is the same for associated and unassociated sources in the different flux bands.
Thus, the correction for the presence of other sources is calculated similarly to the associated sources in Equation \ref{eq:assoc_ev},
but we adjust for the fact that there are fewer unassociated than associated sources, i.e., 
the correction is assumed to be proportionally smaller.
In particular, the number of AGNs among unassociated sources in a certain flux band $\Delta F$ is estimated as

\be
\lb{eq:unassoc_ev}
N_{\rm AGN}^{\rm unass} = \sum_{i \in \rm unass} p^i_{\rm AGN}\,\, - \sum_{i \in \rm ass\,other} p^i_{\rm AGN} \cdot 
\frac{N_{\rm unass}}{N_{\rm ass}}
\ee
where all probabilities and the numbers of sources are computed for sources with flux inside $\Delta F$.
The first term is the sum of AGN-like probabilities among the unassociated sources,
while the second term is the sum of AGN-like probabilities among associated ``other'' sources rescaled by the total number
of unassociated and associated sources in this flux band.
The expected number of pulsars among the unassociated sources is calculated analogously.

We predict that the expected number of pulsars among the unassociated sources in the 3FGL catalog
is $267 \pm 110$, where the range is the envelop of the sums of probabilities in Equation (\ref{eq:unassoc_ev})
for different ML methods (including oversampling) corrected for other sources among the unassociated sources.
The expected number of pulsars among the unassociated sources in the 4FGL catalog corrected for other sources is 
$386 \pm 179$.
These numbers are larger than the number of associated PSRs without missing values (164 in 3FGL and 237 in 4FGL).
Even at the lower range of expected numbers of pulsars among unassociated sources, there are potentially as many pulsars
as there are associated ones.

We note that according to Table \ref{tab:3FGL_prediction}, the number of unassociated 3FGL sources 
with $p_{\rm PSR} > 0.5$ for all four ML algorithms is 96 (83), while there are 332 (309.5) sources with mixed classification,
uncorrected (corrected) for other sources.
The number of sources with mixed classification (309.5 for 3FGL or 475.5 for 4FGL)
is larger than the range of values for the expected number of pulsars calculated for the sum of probabilities 
(220 for 3FGL or 358 for 4FGL).
It means that the decision which sources are considered to be more likely pulsars is more sensitive to the choice of the ML method
and the probability threshold than the expected number of pulsars calculated from on the sum of probabilities.

We also note that the probabilistic classification mostly affects sources with smaller fluxes,
which we illustrate in Figure \ref{fig:unass_vs_ass_frac}, where we show that the ratio of expected number of AGNs and pulsars among unassociated sources computed according to Equation \ref{eq:unassoc_ev} to the number of associated sources decreases as the flux increases.
Negative values (e.g., at high fluxes for AGNs) are due to subtraction of probabilities for the ``other'' associated sources.
%One can see that the fraction of AGNs or pulsars among unassociated sources to the associated ones is higher at smaller fluxes.

\subsection{Latitude and longitude profiles}
\lb{sec:lat-lon-profiles}

\begin{figure*}[h]
\center
%\hspace*{-1cm}
\includegraphics[width=0.45\textwidth]{plots/lat_profile_PSR_3FGL_oversample.pdf}
%\hspace*{-1cm}
\includegraphics[width=0.45\textwidth]{plots/lat_profile_PSR_4FGL_oversample.pdf} \\
\includegraphics[width=0.45\textwidth]{plots/lat_profile_AGN_3FGL_oversample.pdf}
%\hspace*{-1cm}
\includegraphics[width=0.45\textwidth]{plots/lat_profile_AGN_4FGL_oversample.pdf}
\caption{Latitude profiles.}  
\label{fig:lat_profile}
\end{figure*}



\begin{figure*}[h]
\center
%\hspace*{-1cm}
\includegraphics[width=0.45\textwidth]{plots/lon_profile_PSR_3FGL_oversample.pdf}
%\hspace*{-1cm}
\includegraphics[width=0.45\textwidth]{plots/lon_profile_PSR_4FGL_oversample.pdf} \\
\includegraphics[width=0.45\textwidth]{plots/lon_profile_AGN_3FGL_oversample.pdf}
%\hspace*{-1cm}
\includegraphics[width=0.45\textwidth]{plots/lon_profile_AGN_4FGL_oversample.pdf}
\caption{Longitude profiles.}  
\label{fig:lon_profile}
\end{figure*}


In this section we show Galactic latitude and longitude profiles of the distributions of associated and unassociated sources.
In Figure \ref{fig:lat_profile} we present the source counts as a function of ${\rm abs(sin(GLAT))}$,
where we use 20 bins, i.e., each bin corresponds to a solid angle of $4 \pi / 20$. 
Solid blue lines show counts of sources in 3FGL and 4FGL catalogs.
It is interesting to note that the density of associated AGNs is decreasing near the Galactic plane.
The total counts of unassociated sources are shown by red dash-dotted lines.
Green shaded areas show the envelopes of sums of probabilities for AGN- and PSR-like sources for the four 
algorithms without oversampling, while the red shaded areas show the envelope for the four algorithms 
with oversampling.
The classifications of 3FGL sources by \cite{2016ApJ...820....8S} are shown by dashed lines (LR - brown, RF - pink).
%The RF prediction of \cite{2016ApJ...820....8S} lies within the green band, i.e., agrees well with our calculations.
In this section we do not perform a correction for the presence of other sources among the unassociated ones.
The numbers of unassociated sources classified as AGNs and PSRs grow towards the Galactic plane (GP).
Within $\approx 4^\circ.5$ from the GP the expected number of PSRs is about the same as the number of AGNs among unassociated sources (the first data point on the left).
At high latitudes, most of unassociated sources are classified as AGNs.
It is interesting to note, that according to Table \ref{tab:feat_imp}, GLAT is one of the least important features for RF and BDT algorithms.
It can be a posteriori explained by the fact that
 the density of AGNs is such that even in the GP the expected number of AGNs is comparable to the expected number of PSRs.

Gray shaded areas show the sum of the source counts and the expected number of sources for the four algorithm both with and without oversampling.
%The predictions of the 4 algorithms, which we use for classification, added to the counts of associated sources are shown by grey bands.
The average among the eight methods added to the counts of associated sources is shown by solid orange line 
(for AGNs we also show the mean of these points by dotted green line).
We find that the number of associated AGNs is decreasing towards the GP, the expected number of AGNs among unassociated sources is increasing towards the GP, but the sum of the two is relatively uniform as a function of Galactic latitude.


In Figure \ref{fig:lon_profile} we show plots analogous to Figure \ref{fig:lat_profile} for Galactic longitudes.
We note that there is a significant increase in the number of unassociated sources in the 4FGL catalog for $|\ell | \lesssim 50^\circ$.
It leads to a large expected number of pulsars among unassociated sources for these longitudes.
The number of associated AGNs is smaller than average for $|\ell | \lesssim 50^\circ$, while the expected number of AGNs among unassociated sources is larger than average for these longitudes.
The sum of the two is relatively uniform, with a possible overprediction of AGNs in the unassociated sources in the 4FGL catalog for \mbox{$-50^\circ < \ell < 0^\circ$}.










