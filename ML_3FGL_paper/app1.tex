\section{Appendix}

\subsection{Tests with other meta-parameters}

In this appendix we discuss tests of some hyper-parameters, which had a relatively little effect on the 
accuracy of the algorithms. For these tests we use the 3FGL catalog.

LR algorithm has two hyper-parameters regularization and tolerance. 
%These features are the limit up to which one wants to go before accepting a solution. 
As can be seen in figure \ref{fig:LR_tol_reg} the effect on accuracy is less than 1\%. Therefore we used the default values for these parameters. \dima{If we mention the default value, we should say what the default values are}. 
\begin{figure}[h]
%\centerin
\includegraphics[width=\twopicsp\textwidth]{plots/lr_train_reg.pdf}
%\includegraphics[width=\twopicsp\textwidth]{plots/unassoc_complex.pdf}
\caption{Dependence of LR on tolerance and regularization}
\label{fig:LR_tol_reg}
\end{figure}

In Figure \ref{fig:nn_nn} we show the effect of adding the third hidden layer in the NN algorithm.
The difference between the best accuracies with the additional hidden layer is less then 1\%
compare to the NN with two hidden layers (cf. Table \ref{tab:selected_algs}).
\begin{figure}[h]
%\centerin
\includegraphics[width=\twopicsp\textwidth]{plots/nn_2layers_3fgl.pdf}
%\includegraphics[width=\twopicsp\textwidth]{plots/unassoc_complex.pdf}
\caption{Dependence of NN on the second hidden layer, for 10 neurons in the first layer.
\dima{Strictly speaking this is the case of 3 hidden layers.}}
\label{fig:nn_nn}
\end{figure}

At the end we summarize features and their statistics,
which we use for probabilistic classification of sources in the 3FGL and 4FGL catalogs,
in Tables \ref{tab:3FGL_features} and \ref{tab:4FGL_features} respectively. 

\pgfplotstableread[col sep=comma]{data/features/3fglassocfeatures.csv}\tablea
\begin{table}
\resizebox{0.45\textwidth}{!}{
\pgfplotstabletypeset[
columns={Name,Mean,SD,Minimum,Maximum},
column type=c,
string type,
every head row/.style={before row=\toprule,after row=\midrule,},
every last row/.append style={after row={\hline} },
every first column/.style={column type/.add={|}{}},
every last column/.style={column type/.add={}{|}},
columns/Name/.style={column name=Feature Name,string replace*={_}{\textunderscore}},
columns/Mean/.style={column name=Mean,column type=c,numeric type,fixed,precision=2},
columns/SD/.style={column name=Standard Deviation,numeric type,fixed,precision=2},
columns/Minimum/.style={column name=Minimum,numeric type,fixed,precision=2},
columns/Maximum/.style={column name=Maximum,numeric type,fixed,precision=2},
skip rows between index={10}{25}
]{\tablea}
}
\vspace{0.2cm}
\caption{Statistics of features used for probabilistic classification of the 3FGL sources.
\lb{tab:3FGL_features}}
\end{table}

\pgfplotstableread[col sep=comma]{data/features/4fglassocfeatures.csv}\tableaf
\begin{table}
\resizebox{0.45\textwidth}{!}{
\pgfplotstabletypeset[
columns={Name,Mean,SD,Minimum,Maximum},
column type=c,
string type,
every head row/.style={before row=\toprule,after row=\midrule,},
every last row/.append style={after row={\hline} },
every first column/.style={column type/.add={|}{}},
every last column/.style={column type/.add={}{|}},
columns/Name/.style={column name=Feature Name,string replace*={_}{\textunderscore}},
columns/Mean/.style={column name=Mean,column type=c,numeric type,fixed,precision=2},
columns/SD/.style={column name=Standard Deviation,numeric type,fixed,precision=2},
columns/Minimum/.style={column name=Minimum,numeric type,fixed,precision=2},
columns/Maximum/.style={column name=Maximum,numeric type,fixed,precision=2},
%skip rows between index={17}{28}
]{\tableaf}
}
\vspace{0.2cm}
\caption{Statistics of features used for probabilistic classification of the 4FGL sources.
\lb{tab:4FGL_features}}
\end{table}


