% !TEX TS-program = pdflatex
% !TEX encoding = UTF-8 Unicode

% This file is a template using the "beamer" package to create slides for a talk or presentation
% - Talk at a conference/colloquium.
% - Talk length is about 20min.
% - Style is ornate.

% MODIFIED by Jonathan Kew, 2008-07-06
% The header comments and encoding in this file were modified for inclusion with TeXworks.
% The content is otherwise unchanged from the original distributed with the beamer package.

\documentclass{beamer}


% Copyright 2004 by Till Tantau <tantau@users.sourceforge.net>.
%
% In principle, this file can be redistributed and/or modified under
% the terms of the GNU Public License, version 2.
%
% However, this file is supposed to be a template to be modified
% for your own needs. For this reason, if you use this file as a
% template and not specifically distribute it as part of a another
% package/program, I grant the extra permission to freely copy and
% modify this file as you see fit and even to delete this copyright
% notice. 


\mode<presentation>
{
  \usetheme{Warsaw}
  % or ...

  \setbeamercovered{transparent}
  % or whatever (possibly just delete it)
}


\usepackage[english]{babel}
\usepackage[export]{adjustbox}
% or whatever
\usepackage{subcaption}
\usepackage[font={scriptsize}]{caption}
\usepackage[utf8]{inputenc}
%\usepackage{amsmath}
% or whatever
%\usepackage[font=small]{caption}
\usepackage{graphicx,caption}
\usepackage{times}
\usepackage[T1]{fontenc}
\newcommand{\source}[1]{\caption*{\hfill Source: {#1}} }
% Or whatever. Note that the encoding and the font should match. If T1
% does not look nice, try deleting the line with the fontenc.


\title[MMA] % (optional, use only with long paper titles)
{Classification of AGNs and Pulsars using Machine Learning techniques }

%\subtitle
%{Include Only If Paper Has a Subtitle}

\author{Aakash Bhat}%(optional, use only with lots of authors)
% - Give the names in the same order as the appear in the paper.
% - Use the \inst{?} command only if the authors have different
%   affiliation.

\institute % (optional, but mostly needed)
{
 FAU Erlangen-Nurnberg
 % University of Somewhere
  %\and
 % \inst{2}%
 % Department of Theoretical Philosophy\\
 % University of Elsewhere}
% - Use the \inst command only if there are several affiliations.
% - Keep it simple, no one is interested in your street address.
}
\date{Juli 2019} % (optional, should be abbreviation of conference name)
% - Either use conference name or its abbreviation.
% - Not really informative to the audience, more for people (including
%   yourself) who are reading the slides online

\subject{Astronomy}
% This is only inserted into the PDF information catalog. Can be left
% out. 



% If you have a file called "university-logo-filename.xxx", where xxx
% is a graphic format that can be processed by latex or pdflatex,
% resp., then you can add a logo as follows:

% \pgfdeclareimage[height=0.5cm]{university-logo}{university-logo-filename}
% \logo{\pgfuseimage{university-logo}}



% Delete this, if you do not want the table of contents to pop up at
% the beginning of each subsection:
\AtBeginSubsection[]
{
  \begin{frame}<beamer>{Outline}
    \tableofcontents[currentsection,currentsubsection]
  \end{frame}
}


% If you wish to uncover everything in a step-wise fashion, uncomment
% the following command: 

%\beamerdefaultoverlayspecification{<+->}


\begin{document}

\begin{frame}
  \titlepage
\end{frame}

\begin{frame}{Table of contents}
  \tableofcontents
  % You might wish to add the option [pausesections]
\end{frame}


% Structuring a talk is a difficult task and the following structure
% may not be suitable. Here are some rules that apply for this
% solution: 

% - Exactly two or three sections (other than the summary).
% - At *most* three subsections per section.
% - Talk about 30s to 2min per frame. So there should be between about
%   15 and 30 frames, all told.

% - A conference audience is likely to know very little of what you
%   are going to talk about. So *simplify*!
% - In a 20min talk, getting the main ideas across is hard
%   enough. Leave out details, even if it means being less precise than
%   you think necessary.
% - If you omit details that are vital to the proof/implementation,
%   just say so once. Everybody will be happy with that.

\section{Introduction}

\subsection{General Idea}

\begin{frame}

  \begin{itemize}
  \item
Create machine learning algorithms capable of classifying AGNs and Pulsars
  \item
   Use 4-year Fermi LAT catalog to train and test the algorithms
  \item
Apply the algorithms on the newly released 8-year list

  \end{itemize}
\end{frame}


\subsection{Application}

\begin{frame}{Types of sources and algorithms}
 \begin{itemize}
  \item
Use the classified AGNs (BL lacs etc) and Pulsars
  \item
	Find the appropriate "features" to use
  \item
Algorithms include: Random forests, Logistic regression, Neural Networks
  \item
	Estimate performance by focusing on individual models (Number and depth of trees in forest based models, Number of layers in Neural networks)
\end{itemize}
\end{frame}


\section{The Data}

\subsection{AGNs and PSRs}

\begin{frame}{}
\begin{itemize}
 \item
Total of 1905 sources classified
 \item
Features: Flux, uncertainty, Significant curvature, spectral index, Hardness ratios,  Galactic latitude
 \item
70\% training and 30\% testing 
 \item
$hr_{ij}=\frac{EnergyFlux_j - EnergyFlux_i}{EnergyFlux_j + EnergyFlux_i}$
\end{itemize} 

\end{frame}

\begin{frame}{Summary}

  % Keep the summary *very short*.
  \begin{itemize}
  \item
    The \alert{first main message} of your talk in one or two lines.
  \item
    The \alert{second main message} of your talk in one or two lines.
  \item
    Perhaps a \alert{third message}, but not more than that.
  \end{itemize}
  
\end{frame}




\end{document}


